\chapter{Time Waste Versus Empowerment}
\label{rant}

{\it I took a course in speed reading, and read War and Peace in 20
minutes.  It's about Russia}---comedian Woody Allen

{\it I learned very early the difference between knowing the name of
something and knowing something}---Richard Feynman, Nobel laureate in
physics

{\it The main goal [of this course] is self-actualization through the
empowerment of claiming your education}---UCSC (and former UCD)
professor Marc Mangel, in the syllabus for his calculus course

{\it What does this really mean?  Hmm, I've never thought about
that}---UCD PhD student in statistics, in answer to a student who asked
the actual meaning of a very basic concept

{\it You have a PhD in engineering.  You may have forgotten
technical details like $\frac{d}{dt} sin(t) = cos(t)$, but you should at
least understand the concepts of rates of change}---the author, gently
chiding a friend who was having trouble following a simple quantitative
discussion of trends in California's educational system

{\it Give me six hours to chop down a tree and I will spend the first
four sharpening the axe}---Abraham Lincoln

\bigskip

The field of probability and statistics (which, for convenience, I will
refer to simply as ``statistics'' below) impacts many aspects of our
daily lives---business, medicine, the law, government and so on.
Consider just a few examples:

\begin{itemize}

\item The statistical  models used on Wall Street made the ``quants''
(quantitative analysts) rich---but also contributed to the worldwide
financial crash of 2008.

\item In a court trial, large sums of money or the freedom of an accused
may hinge on whether the judge and jury understand some statistical
evidence presented by one side or the other.

\item Wittingly or unconsciously, you are using probability every time
you gamble in a casino---and every time you buy insurance.

\item Statistics is used to determine whether a new medical treatment is
safe/effective for you.

\item Statistics is used to flag possible terrorists---but sometimes
unfairly singling out innocent people while other times missing ones who
really are dangerous.

\end{itemize}

Clearly, statistics {\it matters}.  But it only has value when one
really {\it understands} what it means and what it does.  Indeed,
blindly plugging into statistical formulas can be not only valueless but
in fact highly dangerous, say if a bad drug goes onto the market.

Yet most people view statistics as exactly that---mindless plugging into
boring formulas.  If even the statistics graduate student quoted above
thinks this, how can the students taking the course be blamed
for taking that atititude?

I once had a student who had an unusually good understanding of
probability.  It turned out that this was due to his being highly
successful at playing online poker, winning lots of cash.  No blind
formula-plugging for him!  He really had to {\it understand} how
probability works.  

Statistics is {\it not} just a bunch of formulas.  On the contrary, it
can be mathematically deep, for those who like that kind of thing.
(Much of statistics can be viewed as the Pythagorean Theorem in
n-dimensional or even infinite-dimensional space.)  But the key point is
that {\it anyone} who has taken a calculus course can develop true
understanding of statistics, of real practical value.  As Professor
Mangel says, that's \underline{empowering}.

Statistics is based on probabilistic models.  So, in order to become
effective at data analysis, one must really master the principles of
probability; this is where Lincoln's comment about ``sharpening the
axe'' truly applies.

So as you make your way through this book, always stop to think, ``What
does this equation really mean?  What is its goal?  Why are its
ingredients defined in the way they are?  Might there be a better way?
How does this relate to our daily lives?''  Now THAT is empowering.

