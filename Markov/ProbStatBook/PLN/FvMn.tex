\documentclass[11pt]{article}  % required first line, though can vary;
                               % this says we will use 11-point font,
                               % in the "article" format

% material beginning with the percent sign is commentary, for human
% information purposes, not processed by the LaTeX system

% these \setlength etc. lines concern page layout, amount of paragraph
% indentation etc.; beginners should ignore them (but include them)
\setlength{\oddsidemargin}{0.0in}
\setlength{\evensidemargin}{0.0in}
\setlength{\topmargin}{-0.25in}
\setlength{\headheight}{0in}
\setlength{\headsep}{0in}
\setlength{\textwidth}{6.5in}
\setlength{\textheight}{9.25in}
\setlength{\parindent}{0in}
\setlength{\parskip}{2mm}

\begin{document}  % required; the document starts here

The famous Pythagoren Theorm concerns right triangles.  If the legs of
the triangle have lengths x and y, and the length of the hypotenuse is z,
% the $ delimiter marks the start and end of a mathematical expression
then $z = \sqrt{x^2+y^2}$.

% a blank line means a new paragraph

There are many proofs of this result.

We can get the corresponding angles via inverse trig functions.  For
instance, the angle formed by the sides of length x and z above, which
we'll call $\theta$,   % here's a Greek letter
is

\begin{equation}
\theta = cos^{-1}(\frac{x}{z})  % note the need for braces around the -1
\end{equation}

% the parentheses in the above expression aren't quite big enough;
% instead, we could have the LaTeX code
% \begin{equation}
% \theta = cos^{-1} \left ( \frac{x}{z} \right)  
% \end{equation}
% try it yourself (don't forget to remove the % comment symbols first)

\end{document}  % required; the document ends here

